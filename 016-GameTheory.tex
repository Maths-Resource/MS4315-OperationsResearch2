\documentclass[a4paper,12pt]{article}
%%%%%%%%%%%%%%%%%%%%%%%%%%%%%%%%%%%%%%%%%%%%%%%%%%%%%%%%%%%%%%%%%%%%%%%%%%%%%%%%%%%%%%%%%%%%%%%%%%%%%%%%%%%%%%%%%%%%%%%%%%%%%%%%%%%%%%%%%%%%%%%%%%%%%%%%%%%%%%%%%%%%%%%%%%%%%%%%%%%%%%%%%%%%%%%%%%%%%%%%%%%%%%%%%%%%%%%%%%%%%%%%%%%%%%%%%%%%%%%%%%%%%%%%%%%%
\usepackage{eurosym}
\usepackage{vmargin}
\usepackage{amsmath}
\usepackage{graphics}
\usepackage{epsfig}
\usepackage{enumerate}
\usepackage{subfigure}
\usepackage{fancyhdr}
\usepackage{listings}
\usepackage{framed}
\usepackage{graphicx}
\usepackage{amsmath}
\usepackage{chngpage}
%\usepackage{bigints}


\setcounter{MaxMatrixCols}{10}

\begin{document}
%- https://www3.nd.edu/~tgresik/Econ40050/Exam1Answers.pdf
%======================%
% 2017 Question 92
Provide a short description of each of the following terms.
\begin{enumerate}[(a)]
\item \textbf{ Strategy profile :} A list consisting of one strategy for each player. 
\item \textbf{ Nash equilibrium :} A strategy profile for which each player's strategy is a best response to
the profile of all other players' strategies. 
\item \textbf{ Dominant strategy :} A strategy is dominant for a player if it is the player's best strategy
no matter what strategy choices all other players make. 
\item \textbf{ Dominated strategy :} A strategy is dominated for a player if there is a second strategy that
does at least as well no matter what strategy choices other players make and, for some
profile of strategies for all other players, the second strategy does strictly better. 
\item \textbf{ Mixed strategy :} A mixed strategy involves a player randomizing over 2 or more pure
strategies. 
\end{enumerate}
What is the strategic advantage of using a mixed strategy?
It keeps the other player(s) off balance. They cannot anticipate how you will play the
game. 
%=====================%

The questions in this problem refer to the following game.
\begin{verbatim}
Player
2
 & L& M& R \\
U & 1,2 & 3,5 & 2,1 \\
Player 1 M & 0,4 & 2,1 & 3,0 \\
D & -1,1 & 4,3 & 0,2 \\
\end{verbatim}
\begin{enumerate}[(a)]
\item  Determine if either player has any dominated strategies. If so, identify them.\\
Answer: R for player 2 is dominated by M. For each player 1 strategy, M gives player 2 a higher
payoff than does R. 
\item Does either player have a dominant strategy? Why or why not?\\
Answer: No. For either player to have a dominant strategy, 2 of her 3 strategies would need to be
dominated. 
\item c. Use iterated elimination of dominated strategies to solve this game. Be clear about the order
in which you are eliminating strategies. Also specify whether you are eliminating strictly or
weakly dominated strategies.\\
Answer:\\
1. Eliminate R as above. (Strict) \\
2. In the 3 x 2 game, U strictly dominates M. \\
3. In the 2 x 2 game, M strictly dominates L. \\
4. In the 2 x 1 game, D strictly dominates U.\\
IEDS Solution = (D,M)\\
\item d. Is your solution a Nash equilibrium? Why or why not?
Yes, all IEDS solutions are Nash equilibria. If starting at (U,M), one player had an
alternative strategy that gave her a higher payoff, IEDS would not have eliminated that
alternative strategy. (3)
\end{enumerate}

%============================================% 

\newpage
\begin{enumerate}
    \item %==================================================================================================================================%


	\item Given an LP (the \emph{Primal} problem) we can write a closely
  	related LP, its \emph{Dual}:
  	\begin{eqnarray*}
  		z &=& \max\{c^T x : Ax \le b,  x \in \Bbb r^n, x \ge 0 \}\quad \emph{Primal}\label{eq:lp-primal-1}\\
  		w&=&    \min \{b^T  y : A^T y  \ge c,  y \in \Bbb r^m, y  \ge 0\}. \quad\emph{Dual}\label{eq:lp-dual-1}
  	\end{eqnarray*}
  	\begin{enumerate}\item 
  		Prove the Weak Duality Theorem: for \emph{any} primal feasible
  		point $x$ and \emph{any} dual feasible point $y$, $ b^T y \ge
  		c^T x$.\marks{4\%}
  		
  		\sols{$b^Ty\ge (Ax)^Ty=x^TA^Ty\ge x^Tc$ (as $x,y \ge \bf 0$ and $Ax \ge b$, $A^Ty \ge c$).}
  		\item Show that the Weak Duality Theorem implies that \emph{any}
  		feasible point $y$ for the Dual problem gives an upper bound
  		for the optimal solution $z$ of the original IP, namely $z \le
  		b^T y$.\marks{4\%}
  		
  		\sols{Have $b^T y \ge c^Tx$ for prim feas $x$ and dual feas $y$, so $b^T y \ge \max \{c^Tx| Ax \le b, x\ge 0, x \in \rn\}\ge   \max \{c^Tx| Ax \le b, x\ge 0, x \in \Zn\}     $}
  	\end{enumerate}

%==================================================================================================================================%


\end{document}